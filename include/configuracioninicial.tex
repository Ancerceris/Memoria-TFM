%%%%%%%%%%%%%%%%%%%%%%%%%%%%%%%%%%%%%%%%%%%%%%%%%%%%%%%%%%%%%%%%%%%%%%%%
% Plantilla TFG/TFM
% Escuela Politécnica Superior de la Universidad de Alicante
% Realizado por: Jose Manuel Requena Plens
% Contacto: info@jmrplens.com / Telegram:@jmrplens
%%%%%%%%%%%%%%%%%%%%%%%%%%%%%%%%%%%%%%%%%%%%%%%%%%%%%%%%%%%%%%%%%%%%%%%%

%%%%%%%%%%%%%%%%%%%%%%%%
% FORMATO DEL DOCUMENTO
%%%%%%%%%%%%%%%%%%%%%%%%
% scrbook es la clase de documento
% Si se desea que no haya página en blanco entre capítulos añadir "openany" en los parámetros de la clase. Sino siempre los capítulos empezarán en página impar.
\documentclass[a4paper,11pt,titlepage]{scrbook}
\KOMAoption{toc}{bib,chapterentryfill} % Opciones del índice

% Paquete de formato para scrbook. Con marcas, linea-separador superior e inferior
\usepackage[automark,headsepline,footsepline]{scrlayer-scrpage}
\clearpairofpagestyles		% Borra los estilos por defecto
%%
% Formato y contenido de la información de cabecera y pie de página
%%
% Información de capítulo en cabecera e interno
\ihead{{\color{gray30}\scshape\small\headmark}}	
% Número de página en cabecera y externo
\ohead{\normalfont\pagemark} 
% Número de página en pie de página y externo. Sólo en páginas sin cabecera
\ofoot[\normalfont\pagemark]{}
%% 		
% Edición del contenido de las distintas partes de la cabecera
%%
\renewcommand{\chaptermark}[1]{\markboth{#1}{}} % Capítulo (Solo texto)
\renewcommand{\sectionmark}[1]{\markright{\thesection. #1}} % Sección (Número y texto)
\setkomafont{pagenumber}{} % Número de página (Sin nada añadido)

% Añade al índice y numera hasta la profundidad 4.
% 1:section,2:subsection,3:subsubsection,4:paragraph
\setcounter{tocdepth}{4}
\setcounter{secnumdepth}{4}
% Muestra una regla para comprobar el formato de las páginas
%\usepackage[type=upperleft,showframe,marklength=8mm]{fgruler}
% MÁRGENES DE LAS PÁGINAS
\usepackage[
  inner	=	3.0cm, % Margen interior
  outer	=	2.5cm, % Margen exterior
  top	=	2.5cm, % Margen superior
  bottom=	2.5cm, % Margen inferior
  includeheadfoot, % Incluye cabecera y pie de página en los márgenes
]{geometry}
% Valor de interlineado
\renewcommand{\baselinestretch}{1.0} % 1 línea de interlineado



%%%%%%%%%%%%%%%%%%%%%%%%
% ALGORITMO
%%%%%%%%%%%%%%%%%%%%%%%%
\usepackage{amsmath}
\usepackage{algorithm}
\usepackage{algpseudocode}
\usepackage[framed,numbered]{matlab-prettifier}
\usepackage{listings}

%%%%%%%%%%%%%%%%%%%%%%%% 
% COLORES
%%%%%%%%%%%%%%%%%%%%%%%% 
% Biblioteca de colores
\usepackage{color}
\usepackage[dvipsnames]{xcolor}
% Otros colores definidos por el usuario
\definecolor{gray97}{gray}{.97}
\definecolor{gray75}{gray}{.75}
\definecolor{gray45}{gray}{.45}
\definecolor{gray30}{gray}{.30}
\definecolor{negro}{RGB}{0,0,0}
\definecolor{blanco}{RGB}{255,255,255}
\definecolor{dkgreen}{rgb}{0,.6,0}
\definecolor{dkblue}{rgb}{0,0,.6}
\definecolor{dkyellow}{cmyk}{0,0,.8,.3}
\definecolor{gray}{rgb}{0.5,0.5,0.5}
\definecolor{mauve}{rgb}{0.58,0,0.82}
\definecolor{deepblue}{rgb}{0,0,0.5}
\definecolor{deepred}{rgb}{0.6,0,0}
\definecolor{deepgreen}{rgb}{0,0.5,0}
\definecolor{MyDarkGreen}{rgb}{0.0,0.4,0.0}
\definecolor{bluekeywords}{rgb}{0.13,0.13,1}
\definecolor{greencomments}{rgb}{0,0.5,0}
\definecolor{redstrings}{rgb}{0.9,0,0}

%%%%%%%%%%%%%%%%%%%%%%%%
% BIBLIOGRAFÍA
%%%%%%%%%%%%%%%%%%%%%%%%
\usepackage{apacite} % NORMA APA
\usepackage{natbib}
\usepackage{breakcites}

%%%%%%%%%%%%%%%%%%%%%%%%
% DOCUMENTO EN ESPAÑOL
%%%%%%%%%%%%%%%%%%%%%%%%
\usepackage[base]{babel}
\usepackage{polyglossia}
\setdefaultlanguage{spanish}

\usepackage{longtable,booktabs,array}
\newcolumntype{L}[1]{>{\raggedright\let\newline\\\arraybackslash\hspace{0pt}}m{#1}}
\newcolumntype{C}[1]{>{\centering\let\newline\\\arraybackslash\hspace{0pt}}m{#1}}
\newcolumntype{R}[1]{>{\raggedleft\let\newline\\\arraybackslash\hspace{0pt}}m{#1}}

\ihead{{\color{gray30}\scshape\small\headmark}}	


%%%%%%%%%%%%%%%%%%%%%%%% 
% FIGURAS, TABLAS, ETC 
%%%%%%%%%%%%%%%%%%%%%%%% 
\usepackage{subcaption} % Para poder realizar subfiguras
\usepackage{caption} % Para aumentar las opciones de diseño
% Nombres de figuras, tablas, etc, en negrita la numeración, todo con letra small
\captionsetup{labelfont={bf,small},textfont=small}
% Paquete para modificar los espacios arriba y abajo de una figura o tabla
\usepackage{setspace}
% Define el espacio tanto arriba como abajo de las figuras, tablas
\setlength{\intextsep}{5mm}
% Para ajustar tamaños de texto de toda una tabla o grafica
% Uso: {\scalefont{0.8} \begin{...} \end{...} }
\usepackage{scalefnt}
% Redefine las tablas y figuras para eliminar el '.' entre la numeración y el texto
\renewcommand*{\figureformat}{\figurename~\thefigure}
\renewcommand*{\tableformat}{\tablename~\thetable}

%%%%%%%%%%%%%%%%%%%%%%%% 
% TEXTO
%%%%%%%%%%%%%%%%%%%%%%%%
% Paquete para poder modificar las fuente de texto
\usepackage{xltxtra}
% Cualquier tamaño de texto. Uso: {\fontsize{100pt}{120pt}\selectfont tutexto}
\usepackage{anyfontsize}
% Para modificar parametros del texto.
\usepackage{setspace}
% Paquete para posicionar bloques de texto
\usepackage{textpos}
% Paquete para realizar cajas de texto. 
% Uso: \begin{mdframed}[linecolor=red!100!black] tutexto \end{mdframed}
\usepackage{framed,mdframed}
% Para subrayar. Uso: \hlc[tucolor]{tutexto}
\newcommand{\hlc}[2][yellow]{ {\sethlcolor{#1} \hl{#2}} }

% Para introducir url's con formato. Uso: \url{http://www.google.es}
\usepackage{url}
% Amplia muchas funciones graficas de latex
\usepackage{graphicx}
% Paquete que añade el hipervinculo en referencias dentro del documento, indice, etc
% Se define sin bordes alrededor. Uso: \ref{tulabel}
\usepackage[pdfborder={000}]{hyperref}
\usepackage{float}
\usepackage{verbatim}
% Paquete para condicionales avanzados
\usepackage{xstring,xifthen}
% Paquete para realizar calculos en el código
\usepackage{calc}
% Para rotar tablas o figuras o su contenido
\usepackage{rotating} 
\usepackage{tikz,tikzpagenodes}
\usetikzlibrary{tikzmark,calc,shapes.geometric,arrows,backgrounds,shadings,shapes.arrows,shapes.symbols,shadows,positioning,fit,automata,patterns,intersections}


\addto\captionsspanish{\def\tablename{Tabla}} % para que escriba "Tabla" en lugar de "Cuadro"
\addto\captionsspanish{%
  \renewcommand{\listtablename}{Índice de Tablas} % Cambiar el título del índice de cuadros
}
%%%%%%%%%%%%%%%%%%%%%%%% 
% GLOSARIOS
%%%%%%%%%%%%%%%%%%%%%%%%
\usepackage[acronym,nonumberlist,toc]{glossaries}
\usepackage{glossary-superragged}
\newglossarystyle{modsuper}{%
  \setglossarystyle{super}%
  \renewcommand{\glsgroupskip}{}
}
\renewcommand{\glsnamefont}[1]{\textbf{#1}}


%%%%%%%%%%%%%%%%%%%%%%%% 
% COMANDOS AÑADIDOS
%%%%%%%%%%%%%%%%%%%%%%%%
% Para mostrar la fecha actual (mes año) con \Hoy
\newcommand{\MES}{%
  \ifcase\month% 0
    \or Enero% 1
    \or Febrero% 2
    \or Marzo% 3
    \or Abril% 4
    \or Mayo% 5
    \or Junio% 6
    \or Julio% 7
    \or Agosto% 8
    \or Septiembre% 9
    \or Octubre% 10
    \or Noviembre% 11
    \or Diciembre% 12
  \fi}
\newcommand{\ANYO}{\number\year}
\newcommand{\Hoy}{\MES\ \ANYO}

%%%%%%%%%%%%%%%%%%%%%%%% 
% MATEMÁTICAS
%%%%%%%%%%%%%%%%%%%%%%%%
\usepackage{mathtools,amsthm,amsfonts,amssymb,bm,mathrsfs,nicefrac,upgreek,bigints} 
% Comando para añadir información de variables a las ecuaciones
% Uso: \begin{condiciones}[donde:] ....... \end{condiciones}
\newenvironment{condiciones}[1][2]
  {%
   #1\tabularx{\textwidth-\widthof{#1}}[t]{
     >{$}l<{$} @{}>{${}}c<{{}$}@{} >{\raggedright\arraybackslash}X
   }%
  }
  {\endtabularx\\[\belowdisplayskip]}

%%%%%
% DEFINICION DE CONCEPTOS
%%%%
% Uso ejemplo: \begin{ejemplo} tucontenido \end{ejemplo} 
\newtheorem{teorema}{Teorema}[chapter]
\newtheorem{ejemplo}{Ejemplo}[chapter]
\newtheorem{definicion}{Definición}[chapter]

\usepackage{enumerate}