%%%%%%%%%%%%%%%%%%%%%%%%%%%%%%%%%%%%%%%%%%%%%%%%%%%%%%%%%%%%%%%%%%%%%%%%
% Plantilla TFG/TFM
% Escuela Politécnica Superior de la Universidad de Alicante
% Realizado por: Jose Manuel Requena Plens
% Contacto: info@jmrplens.com / Telegram:@jmrplens
%%%%%%%%%%%%%%%%%%%%%%%%%%%%%%%%%%%%%%%%%%%%%%%%%%%%%%%%%%%%%%%%%%%%%%%%

\chapter{Metodología (Con ejemplos de figuras)}
\label{metodologia}

\section{Inserción de figuras}

Las figuras son un caso un poco especial ya que \LaTeX~busca el mejor lugar para ponerlas, no siendo necesariamente el lugar donde está la referencia. Por ello es importante añadirle un ``caption'' y un ``label'' para poder hacer referencia a ellas en el párrafo correspondiente. Nosotros ponemos la referencia a la figura \ref{multiimagen} que está en la página \pageref{multiimagen}, justo aquí debajo, pero \LaTeX ~puede que la ubique en otro lugar. (observa el código \LaTeX~ de este párrafo para observar como se realizan las referencias. Estos detalles también se aplican a tablas y otros objetos).

\begin{table}[h]
\centering
\begin{tabular}{ccc}
\includegraphics[scale=0.2]{archivos/130} & \includegraphics[scale=0.2]{archivos/160} & \includegraphics[scale=0.2]{archivos/190} \\
$Dist=1m \; ; \; \phi=30º$  & $Dist=1m \; ; \; \phi=60º$  & $Dist=1m \; ; \; \phi=90º$  \\
\includegraphics[scale=0.2]{archivos/230} & \includegraphics[scale=0.2]{archivos/260} & \includegraphics[scale=0.2]{archivos/290} \\
$Dist=2m \; ; \; \phi=30º$  & $Dist=2m \; ; \; \phi=60º$  & $Dist=2m \; ; \; \phi=90º$  \\
\includegraphics[scale=0.2]{archivos/330} & \includegraphics[scale=0.2]{archivos/360} & \includegraphics[scale=0.2]{archivos/390} \\
$Dist=3m \; ; \; \phi=30º$  & $Dist=3m \; ; \; \phi=60º$  & $Dist=3m \; ; \; \phi=90º$ \\
\end{tabular}
\caption{Esta es una tabla con múltiples imágenes. Útil cuando se deben mostrar varias juntas.}
\label{multiimagen} % 
\end{table}
