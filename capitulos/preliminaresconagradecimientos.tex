%%%%%%%%%%%%%%%%%%%%%%%%%%%%%%%%%%%%%%%%%%%%%%%%%%%%%%%%%%%%%%%%%%%%%%%%
% Plantilla TFG/TFM
% Escuela Politécnica Superior de la Universidad de Alicante
% Realizado por: Jose Manuel Requena Plens
% Contacto: info@jmrplens.com / Telegram:@jmrplens
%%%%%%%%%%%%%%%%%%%%%%%%%%%%%%%%%%%%%%%%%%%%%%%%%%%%%%%%%%%%%%%%%%%%%%%%

\chapter*{Abstract}
\thispagestyle{empty}

In recent years, technological advancements have found applications far beyond the media sphere, extending into fields such as ornithology, animal behavior studies, and ecological monitoring. This master's thesis explores novel approaches to analyzing the movement patterns, behavioral characteristics, and habitat preferences of various bird species by leveraging video surveillance data and state-of-the-art deep learning techniques.

The study involves the integration of computer vision, neural network architectures, and data preprocessing tools within the Python programming environment. Key libraries employed in the analysis include \textbf{TensorFlow}, \textbf{PyTorch}, \textbf{OpenCV}, \textbf{scikit-learn}, and \textbf{pandas}, which facilitate efficient data handling, video analysis, and model training.

As part of the project, a web-based visualization platform will be developed using \textbf{CesiumJS} and \textbf{Flask} allowing for an interactive and geospatial representation of bird movement data. The platform will provide users with an intuitive interface to explore behavioral statistics, track migratory patterns, and analyze habitat usage over time.

This research aims to contribute to the growing body of interdisciplinary work at the intersection of artificial intelligence and wildlife monitoring, offering practical tools for researchers and conservationists.


\cleardoublepage %salta a nueva página impar
\chapter*{Acknowledgements}


\thispagestyle{empty}
\vspace{1cm}

I would like to express my sincere gratitude to all those who contributed to the realization of this work. First and foremost, I am deeply thankful to my academic tutor, \textbf{Jose García Rodríguez}, and his research assistant, \textbf{Javier Rodríguez Juan}, for their invaluable guidance, support, and encouragement throughout this project.

I am especially grateful for the opportunity to participate in this particular research project focused on the study of bird behavior. As someone who has always had a deep appreciation for animals and wildlife research, being part of a scientific endeavor that combines technology with the natural world has been both personally meaningful and intellectually fulfilling.

This research would not have been possible without the unwavering support of my family, \textbf{Abby} and \textbf{Amigo}, whose presence has been a constant source of strength and motivation. I would also like to extend heartfelt thanks to the individuals who supported me during my time in Spain — your kindness and encouragement played a significant role in my personal and academic development.

Special thanks go to my close friends from my previous university, whose friendship and belief in me helped sustain my motivation. In particular, I feel deeply indebted to \textbf{Giancarlo}, \textbf{Gabriel}, \textbf{Pancho}, and all those who stood by me and inspired me to pursue this path.


\cleardoublepage %salta a nueva página impar
