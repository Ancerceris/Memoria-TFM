%%%%%%%%%%%%%%%%%%%%%%%%%%%%%%%%%%%%%%%%%%%%%%%%%%%%%%%%%%%%%%%%%%%%%%%%
% Plantilla TFG/TFM
% Escuela Politécnica Superior de la Universidad de Alicante
% Realizado por: Jose Manuel Requena Plens
% Contacto: info@jmrplens.com / Telegram:@jmrplens
%%%%%%%%%%%%%%%%%%%%%%%%%%%%%%%%%%%%%%%%%%%%%%%%%%%%%%%%%%%%%%%%%%%%%%%%

\chapter{Objetivos (Con ejemplos de tablas)}
\label{objetivos}

\section{Tablas}

\begin{table}[h]
	\centering
	\begin{tabular}{lllll}
		&columna A&columna B&columna C\\
		\hline
		fila 1&fila 1, columna A & fila 1, columna B & fila 1, columna C\\
		fila 2&fila 2, columna A & fila 2, columna B & fila 2, columna C\\
		fila 3&fila 3, columna A & fila 3, columna B & fila 3, columna C\\ \hline
	\end{tabular}
	\caption{Ejemplo de tabla.}
	\label{tabladeejemplo}
\end{table}

Existe la posibilidad de forzar que las tablas, figuras u otros objetos aparezcan en la zona del texto que se desea aunque en ocasiones puede dejar grandes espacios en blanco. El comando a utilizar es:

\begin{table}[ht]
	\centering
	\begin{tabular}{|C{2cm}|C{2cm}|C{2cm}|C{2cm}|}
		\hline
		\multicolumn{4}{|c|}{\textbf{\begin{tabular}[c]{@{}c@{}}FUENTE: TRÁFICO RODADO\\ HORARIO: TARDE\end{tabular}}} \\ \hline
		\textbf{dB(A)} & \textbf{Población expuesta tarde} & \textbf{\%} & \textbf{\scriptsize{CENTENAS}} \\ \hline
		\textbf{\textgreater70} & 0 & 0,000 & 0 \\ \hline
		\textbf{65 - 70} & 348,9 & 9,792 & 3 \\ \hline
		\textbf{60 - 65} & 1594,7 & 44,757 & 16 \\ \hline
		\textbf{55 - 60} & 322,1 & 9,040 & 3 \\ \hline
		\textbf{50 - 55} & 0 & 0,000 & 0 \\ \hline
		\textbf{\textgreater50} & 1297,3 & 36,410 & 13 \\ \hline
		\textbf{TOTAL} & 3563 & 100 & 35 \\ \hline
	\end{tabular}
	\label{my-label}
\end{table}	


\section{Otros diseños de tablas}

% EJEMPLO 1
\begin{table}[ht]
	\centering
	{\scalefont{0.9}
	\begin{tabular}{@{}lcc@{}}
	\toprule
	Modelo			& 	15LEX1600Nd	&	15P1000Fe V2	 	\\ \midrule
	fs ($Hz$)		& 	41          & 	45           	\\
	Re ($ohm$)		& 5.5         	& 5.2         	 	\\
	Le ($\mu H$)	& 1600        	& 1500         		\\
	Bl ($N/A$)		& 25.7        	& 27.4         		\\
	M\textsubscript{MS} ($g$)		& 175	& 157		\\
	C\textsubscript{MS} ($\mu m/N$)	& 84		& 78			\\
	R\textsubscript{MS} ($kg/s$)		& 6.8 	& 7.6   	 	\\
	d ($cm$)		& 33.5			& 33           		\\
	Vas ($dm^3$)   	& 91          	& 80.7         		\\
	$Q_\text{TS}$  	& 0.36        	& 0.30         		\\
	$Q_\text{MS}$  	& 6.6         	& 5.9          		\\
	$Q_\text{ES}$  	& 0.38        	& 0.31         		\\
	Sens (dB @ 2.83V/1m) & 96      	& 98           		\\
	$\eta$          & 1.7\%       	& 2.4\%        		\\
	Sd ($cm^2$)   	& 880         	& 855          		\\ \bottomrule
	\end{tabular}
	}
	\caption{Parámetros de los altavoces elegidos de la marca Beyma\textsuperscript{\tiny\textregistered}.}
	\label{tablaparametros}
\end{table}






