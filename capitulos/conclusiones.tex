%%%%%%%%%%%%%%%%%%%%%%%%%%%%%%%%%%%%%%%%%%%%%%%%%%%%%%%%%%%%%%%%%%%%%%%%
% Plantilla TFG/TFM
% Escuela Politécnica Superior de la Universidad de Alicante
% Realizado por: Jose Manuel Requena Plens
% Contacto: info@jmrplens.com / Telegram:@jmrplens
%%%%%%%%%%%%%%%%%%%%%%%%%%%%%%%%%%%%%%%%%%%%%%%%%%%%%%%%%%%%%%%%%%%%%%%%

\chapter{Conclusiones (Con ejemplos de matemáticas)}
\label{conclusiones}

\section{Matemáticas}

En \LaTeX~se pueden mostrar ecuaciones de varias formas, cada una de ellas para un fin concreto.
\par Antes de ver algunas de estas formas hay que conocer cómo se escriben fórmulas matemáticas en \LaTeX. Una fuente de información completa es la siguiente: \url{https://en.wikibooks.org/wiki/LaTeX/Mathematics}. También existen herramientas online que permiten realizar ecuaciones mediante interfaz gráfica como \url{http://www.hostmath.com/}, \url{https://www.mathcha.io/editor} o \url{https://www.latex4technics.com/}
\vspace{1em}
\noindent\hrule
\vspace{1em}

\begin{equation}
  \nabla\times{\mathbf H}=\left[\frac{1}{r}\frac{\partial}{\partial
        r}(rH_\theta)-\frac{1}{r}\frac{\partial
        H_r}{\partial\theta}\right]{\hat{\mathbf z}}
        \label{ecuacion}
\end{equation}
\vspace{1em}
\noindent\hrule
\vspace{1em}
Si es necesario agrupar varias ecuaciones en un mismo índice se puede escribir del siguiente modo:

\begin{subequations}
  \begin{eqnarray}
    {\mathbf E}&=&E_z(r,\theta)\hat{\mathbf z}\label{ecu1} \\
    {\mathbf H}&=&H_r(r,\theta))\hat{ \mathbf r}+H_\theta(r,\theta)\hat{\bm
      \theta}\label{ecu2}
  \end{eqnarray}
\end{subequations}
\vspace{1em}
\noindent\hrule
\vspace{1em}

